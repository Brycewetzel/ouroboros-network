% -*- fill-column: 80 -*-

\documentclass[11pt,a4paper]{article}

\usepackage[margin=2.5cm]{geometry}
\usepackage[dvipsnames]{xcolor}
\usepackage{todonotes}
\usepackage{microtype}
\usepackage{amssymb,amsmath}
\usepackage{mathpazo}
\usepackage{longtable,booktabs}
\usepackage{dcolumn}
\usepackage{graphicx}
\usepackage{natbib}
\usepackage{hyperref}
\usepackage[capitalise,noabbrev,nameinlink]{cleveref}
\hypersetup{
  pdftitle={Storing the Cardano ledger state on disk: analysis and design options},
  pdfborder={0 0 0},
  breaklinks=true
}

\begin{document}

\title{Storing the Cardano ledger state on disk: \\
       analysis and design options \\
       {\large \sc An IOHK technical report}
  }
\date{Version 0.9, April 2021}
\author{Douglas Wilson     \\ {\small \texttt{douglas@well-typed.com}} \\
   \and Duncan Coutts      \\ {\small \texttt{duncan@well-typed.com}} \\
                              {\small \texttt{duncan.coutts@iohk.io}}
   }

\maketitle

%\listoftodos

\section{Introduction}
\label{introduction}

The project is intended to solve the following problem: the Cardano node keeps
its ledger state within memory (RAM) and as Cardano scales up this will not be
sustainable because the ledger state will eventually grow too big.

The solution that this project is focused on is to move the bulk of the ledger
state from memory to disk. This will involved developing and then integrating
new infrastructure in the Cardano node (ledger and consensus layers) to allow
large parts of the ledger state to be kept on disk rather than in memory.

This document is intended to capture the business and technical requirements,
to present a small set of candidate solutions, and to evaluate and justify a
preferred solution among those candidates. That solution should be justified in
complexity by the captured requirements.

\tableofcontents

\section{Summary}
\label{summary}

Memory (RAM) is very fast but memory space is relatively small and expensive.
Disks -- even SSDs -- are relatively slow, but disk space is relatively
plentiful and cheap.

Thus the challenge when adapting a design from memory to disk is to maintain
adequate performance. For blockchains generally, and Cardano specifically, this
is not a trivial problem. For example for a long time Ethereum node performance
was bottlenecked on disk I/O performance. We must consider performance in the
design analysis or we will face the same problem.

Best case estimates indicate that the stretch goal of 200 TPS is \emph{very}
hard to achieve if one also wants the time to synchronise the chain to be
reasonable (e.g. the first time Daedalus is used). Even threshold or mid
targets of 20 and 50 TPS will be a challenge, while keeping sync times
reasonable.

The reason for this is clear: if we were to aim for syncing being 1000 times
faster than real time, then a year's worth of chain data would take just under
9 hours to validate. End users would have to wait 9 hours for each year that
the chain had been operating at this rate. Arguably, even this is unreasonably
slow, and yet note that this already requires syncing to be 1000 times faster.
So if our target were 200 TPS, then the requirement for syncing would be
200,000 TPS. It is intuitively clear that 20,000 or 200,000 TPS is a hard
target indeed.

Overall this points to the next major scaling bottleneck being synchronisation
performance. Though it is out of scope for this project, it will likely be
worth developing solutions that do not require all nodes (e.g. Daedalus nodes)
to download and validate the entire chain.

The simplified optimistic estimate is as follows:
\begin{itemize}
\item Assume the 200 TPS stretch goal.
\item Assume the typical 2 inputs and 2 outputs per tx.
\item Assume the UTxO is the only part of the ledger state of interest. This is
      a simplifying assumption. In reality there are other parts of the ledger
      state that will make these estimates worse.
\item Assume the UTxO mostly does not fit in memory.
\item Assume the UTxO read access pattern has poor temporal and physical
      locality, and thus each lookup will typically require at least one
      ``random'' disk read.
\item Assume that writes to the database are able to be efficiently dispatched
        in batches such that they are insignificant in cost relative to reads.
        Note that this is a rosy assumption that may be mostly true for LSM trees,
        but would not be true for many other data structures (e.g. B+ trees)
\item Assume a DB read amplification factor of 1.5. (This is a rosy assumption.)
\item Thus 200 TPS translates to $200 * 2 * 1.5 = 600$ disk I/O operations per
      second (IOPS).
\item Assume a mid-range SSD rated at 10,000 IOPS for random reads at queue
      depth 1, and 100,000 IOPS for random reads at queue depth 32.
\item Assume that we can fully utilise the parallel performance of the SSD,
      i.e. use queue depth 32.
\item The ratio of SSD performance IOPS (100,000) to live system IOPS (600)
      gives the sync speed ratio, i.e. the factor that syncing the chain would
      be compared to real time. This is a factor of $100,000 / 600 = 167$ in
      our example.
\item Thus for a chain growing at 200 TPS for a year, the time to sync that
      chain would be $1/167$ of a year, which is 52.5 hours, more than two days.
\end{itemize}

As we can see in this estimate, the chain sync times are not reasonable.
The example illustrates that the crucial factors are:
\begin{enumerate}
\item The target TPS
\item The read amplification factor
\item a database that can efficiently batch writes
\item The SSD random read performance
\end{enumerate}
Thus if we reduce the target TPS by a factor of 10, we reduce the sync time bound
correspondingly. With 20 rather than 200 TPS we could expect at best 5 hours of syncing
to catch up a year of the chain.

A read amplification factor of only 1.5 is itself a challenge, and requires a
good database choice.

Modern SSDs IOPS for random read range from 100,000 to 1,000,000 for the
extreme end of the consumer market. Achieving these levels of performance
requires utilising parallel I/O. If only serial I/O is used, one is limited to
the approximately 10,000 -- 20,000 IOPS. Using parallel I/O requires a more
complex design and developing other additional software infrastructure.

So as we can see, even with 20 TPS, to achieve reasonable sync times will
require a sophisticated choice of database, and the development effort needed
to take advantage of parallel I/O. Or it requires relaxing the assumption that
most of the ledger state does not fit in memory: allowing users with lots of
memory to sync quickly, while other users sync slowly.

The recommended development path is to assume that initially (e.g. 12 months)
the TPS will remain relatively low, and that the size of the ledger state will
remain only somewhat bigger than memory. In this case it may be possible to
develop a solution that does not initially use a very sophisticated database
and does not use parallel I/O, but follows a design that can be extended to do
so.

\section{Requirements and targets}
\label{requirements}

\subsection{Transaction Rate (TPS)}

The transaction rate is the average number of \emph{transactions per second}
(TPS). It is a measure of the rate that data is added to a blockchain and is
often used to compare different systems. It is a crude measure because it does
not take account of the transaction size, which can vary wildly. Nevertheless,
because it is the measure used to compare systems, our targets are also
expressed in TPS.

Our targets are

\begin{center}
\begin{tabular}[]{lr}
  Target    & TPS \\
  \toprule
  Threshold &  20 \\
  Middle    &  50 \\
  Stretch   & 200
\end{tabular}
\end{center}
%
For comparison, these are the current transaction rates of Cardano and
comparable systems%
\footnote{\url{https://blockchair.com/bitcoin/charts/transactions-per-second}}%
\footnote{\url{https://blockchair.com/ethereum/charts/transactions-per-second}}.

\begin{center}
\begin{tabular}[]{lr}
  System    & TPS (approx) \\
  \toprule
  Bitcoin                  & 4   \\
  Ethereum, peak           & 17  \\
  Cardano, mainnet typical & 1   \\
  Cardano, mainnet max     & 7   \\
  Cardano, benchmarks max  & \textgreater 50   \\
\end{tabular}
\end{center}

It is out of scope for this storage project to demonstrate the node running at
50 or 200 TPS. It is within scope to demonstrate that the storage system would
not be a bottleneck that would prevent the node running at 50 or 200 TPS.

\subsection{Ledger state size}

Cardano currently has approximately 2 million UTxO entries. For comparison,
Bitcoin currently has approximately 75 million UTxO entries%
\footnote{\url{https://www.blockchain.com/charts/utxo-count}}.

Cardano also has delegation. The current ratio between UTxO entries
and registered stake keys is 5:1. We will assume this 5:1 persists.

Our targets for UTxO sizes are

\begin{center}
\begin{tabular}[]{lrr}
  Target (millions) & UTxO entries & Stake keys \\
  \toprule
  Threshold &  10  &  2 \\
  Middle    &  50  & 10 \\
  Stretch   & 100  & 20
\end{tabular}
\end{center}

We assume the number of registered stake pools will remain in the low thousands.

\subsection{Cardano Node Resource Requirements}

The resource requirements of the Cardano node should not significantly increase,
except perhaps for disk space, and in particular the memory reqirements should
decrease.  The resource requirements currently are:

\paragraph{current cardano requirements}
CPU: ?
Memory: 8GB
Disk: ?
\todo{fill in current resource requirements}


Once the solution is implemented we would expect the Memory requirements to drop
to 4GB and for the Disk space requirements to no more than double.

\subsection{Performance Requirements}
\label{Performance Requirements}
As described in \todo{link}, we expect the sync time to be the most important performance measure.

1 day to sync 1 year of a 20 T/s blockchain

\subsection{Functional Requirements}

The ledger state consists of small complicated parts and large simple parts. The
large simple parts are:

UTxO: A set of all unspent transaction outputs
Stake Delegation: \todo {stake delegation}
Stake Aggregation: \todo {stake aggregation}

The solution must store all of the UTxO and Stake Delegation on disk.  The
solution should store Stake Aggregation on disk, although this will likely be
more difficult, and there is less urgency here. If this is not delivered there
must be a clear path towards it.

\section{Related components}
\label{components}

\subsection{Ledger}

Accesses to UTxO will need to adjust to new API
For all ledger types

\subsection{Consensus}

Accesses to UTxO will need to adjust to new API
Integrate with consensus tests


\subsection{DBSync + Cardano API clients}

\section{Technical constraints}
\label{constraints}

\subsubsection{workload}
\label{workload}
The workload, for the UTxO storage in particular, is write heavy. In the usual
case, we would expect three operations on an UTxO during it's lifetime:

\begin{enumerate}
\item One INSERT operation when the UTxO is created
\item One LOOKUP operation during the validation of the block in which the UTxO is spent
\item One DELETE operation when that block is added to a chain
\end{enumerate}

Note that some UTxOs will be read more often, if validation of a block fails, or
if a block is validated multiple times. From Alonzo era forward, an UTxO may be
present in multiple blocks if it is part of a script transaction that fails to
validate. Nevertheless, we expect the vast majority of UTxOs to be validated
once, and this applies moreso while syncing, where we do not expect validation
to fail. By definition, UTxOs will be written at most twice.

Despite the 2:1 write:read ratio elaborated above, we are likely to be primarily
bottle-necked the syncing operation by read performance. This is because we
expect to be able to easily batch many writes together. Batching reads is more
difficult, with the lack of locality in our data \ref{Locality}.

\subsection{integrate with io-sim}
\label{io-sim}
The Ledger and Consensus codebases have achieved impressive quality
and reliability, in no small part due to their use of the io-sim
Haskell library to test against a broad array of failure modes. This
relies upon integrating at the Haskell source level.

The solution should similarly demonstrate it's correctness with an io-sim test
suite. This is not a strict requirement, however it is difficult to see how one
could be confident in the correctness of the solution without it.

\subsection{minimize changes to STS}

The Cardano Ledger rules are implemented are formulated as executable
specifications (STS). \todo{right? citation?}. While the Byron ledger rules have
a separate implementation that is verified against the executable specification,
Shelley (and the following era's) use the executable specification directly.
This was a pragmatic choice at the time it was made, however it does introduce
tensions. The Ledger team are limited in their ability to modifying the
implementation of those rules to meet business requirements, because changing
the executable specification risks introducing unsoundness to the ledger rules.

The solution should minimise disruption to the implementation of the ledger
rules. This may require us to separate the implementation of the Shelley (and
forward) ledger rules from their executable specification.

\subsection{Batching IO}
\label{Batching IO}
Modern SSDs achieve IO-ops/S in the hundreds of thousands operating concurrently.
Typically they will server up to 32 operations at the same time. For software to
saturate the bandwidth of the disk, that software must be prepared to issue many
operations simultaneously. There is a trade-off here between performance and
complexity. To saturate the bandwidth of the disk, as we expect to be necessary
to achieve reasonable sync times \todo{add ref}, will require more complexity in
the solution.

It is very likely that a solution required to perform well with a growing
blockchain will need to exploit Batching IO to achieve this.

\subsection{Operations}
\label{Operations}
Here we describe the operations the Ledger state on-disk data store should
support.

\begin{enumerate}
  \item INSERT
        The data store must be able to store data against a key. There will be
        some flexibility in how updates to data are applied. Stake delegation
        and aggregation may benefit from some more specialised operations,
        though likely a simple INSERT with the new value would suffice.
  \item LOOKUP
        The data store must be able to retrieve the data previously stored
        against a key
  \item DELETE
        The data store must be able to delete a key and the data stored against
        it. This is important for storing UTxOs, because the growth of unspent
        transaction outputs grows much more slowly than the total historical
        transaction outputs, and so deleting spent transactions will lighten the
        load on the data store manyfold.
  \item ROLLBACK
        \label{ROLLBACK}
        The consensus layer will rollback ledger state whenever it switches
        chain forks. The data store must support this in some way.
  \item

\end{enumerate}

\subsubsection{Locality}
\label{Locality}
Many data structures exploit locality in their data. For example, a time series database will store together values that are adjacent in time. In this way, queries which inspect values that are close in time (most all queries to a time series database!) are able to read a single disk page, and obtain many values of interest.

There are two obvious dimensions in locality for UTxOs: Time of creation and Id (A hash).

We can expect no locality in the ID. IDs are effectively random, and there is no reason to think that two UTxOs with similar IDs would be accessed together.

Less obviously, indeed empirically, there is little temporal locality in blockchains \todo{cite}.
Thus we have a pessimistic towards applying a caching to the Ledger State data structure. Nevertheless, see \ref{Caching Layer}

\section{Design options}
\label{options}

It's clear we will need to implement an disk Key-Value Store capable of serving
the ledger state. The natural choice in the literature for this problem is the Log Structured Merge Tree \todo{cite LSM}. \todo{describe LSM?, liberate a diagram from somewhere?}

\subsection{Off-the-shelf Key-Value Store}

\todo[inline]{
High chance of ``mostly working''
Unlikely to be as reliable as the rest of cardano node
Will likely increase deployment complexity (logging/monitoring/backups/etc)
Good for benchmarking
Add that Cardano used to use RocksDB, and it was problematic
}

One might ask why not use an existing on-disk Key-Value store, such as
RocksDB \todo{cite RocksDB}, or \todo{cite sqlite}? There are two
reasons that we might expect this to be unsatisfactory:

\subsubsection{io-sim testing}
The tooling described \ref{io-sim} would be difficult or impossible to use with
an Off-the-shelf Key-Value Store.

\subsection{Existing Haskell Libraries}

There are no mature LSM libraries on hackage. There does exist a B+-tree
implementation \todo{cite https://hackage.haskell.org/package/haskey-btree},
which we could likely turn to our needs.

A B+-tree will not scale as far as a LSM, but for existing transaction rates it
may suffice.

Implementing the On-disk Ledger State with this library would allow us to ship a
solution with significantly less effort than writing an On-Disk data structure
from scratch. We would also expect less risk of delivery failure. We would be
able to leverage the io-sim library to give confidence of quality.

Note however that a solution based on B+-trees is unlikely to satisfy the
requirements \ref{Performance Requirements}.

\subsection{Bespoke LSM Tree}

Since no sufficient on disk data structure exists, we could create one. A data
structure in the literature, the LSM Tree \cite{monkey}, is appropriate. It offers:

\begin{enumerate}
  \item Append only writes to disk
        This is critical for good write performance. There no need to ``sync''
        with the disk, meaning the program never needs to wait for the disk to be ready.
  \item A snapshotting mechanism
        In normal operation the Cardano Node frequently needs to roll back the
        ledger state, see \ref{ROLLBACK}.
  \item Workload
        There is a  well specified recipe in \cite{monkey} for tuning  the data
        structure to various workloads. Our workload \ref{workload} is quite unusual. We
        would be able to tune the LSM tree to match that workload, and moreover, if the
        workload changes in the future as Cardano evolves, we will have the opportunity
        to revisit these decisions without reworking the entire data structure.
\end{enumerate}

We would plan to follow the blueprint laid out by \cite{monkey}. This approach
does carry a higher risk of failure, as a greenfield project with many unknowns.

\subsection{Batched IO}
As described in \ref{Batching IO}, we will have the option to use Batching IO,
at the cost of additional complexity in the implementation. While a scalable
solution will likely need to do this, it is likely wise to start with a simpler,
slower IO model, and then add Batching IO once we have demonstrated the
correctness of that simpler implementation.

\subsection{Interface Style}

The decision for which on disk data structure to use, and how to implement it,
is largely orthogonal to the decisions around what the interface to that data
structure should look like.

A naive ``CRUD'' interface to the data structure would be simpler to work with,
and would perhaps ease integration of the data structure into the Related Components
codebases. However this would impede our exploitation of Batched IO. \ref{Batching IO}.

Instead, we recommend an interface which allows the data structure to read/write
many values in parallel. Note that committing to this interface does not require
us to implementing parallel operations immediately. A parallel interface could
be implemented using an internal ``CRUD'' interface.

e.g.
readUTxOs :: [TxIn] -> (Map TxIn TxOut -> m a) -> m a
writeUTxOs :: [(TxIn, Maybe TxOut)] -> m ()

\todo{code? delete or format properly}

Note: We intend to elaborate on this interface, and propose a design, in a
following document.

\subsection{Caching Layer}
\label{Caching Layer}

Although we expect caching to perform poorly over this data structure, due to the
lack of locality \ref{Locality}, perhaps a poor cache would be significantly
better than nothing.

One option to mitigate long sync times may be to allow users to add a large
in-memory cache over the data structure while syncing. In this way, users could
sync on a large cloud machine (Say 64GB RAM), save their state to disk, then run
their node on a smaller cloud machine using that synced state.


\subsection{Hybrid options}

We are not bound to a single choice of those above. Although we will recommend that
a bespoke LSM tree should be the long term goal, it may be the case that delivery of a
working solution, that does not meet the performance requirements as such, is
the best first step.

We should aim to specify an interface with can be served by any of the three
options above.

\section{Preferred option}
\label{preferred}

Our preferred option is a hybrid, seeking to fail early, minimize delivery risk,
deliver some improvement early, and giving the option of pausing work before
completion. We will design the API early and implement several iterations of the
data structure, each with more complexity and better performance than the
previous.

Firstly, we would deliver a Draft API for agreement with stakeholders (Ledger
and Consensus). This would be prioritised immediately. Once agreed, we would
deliver a trivial implementation of a data store that was still entirely in
memory, using the existing algorithms, but which is interacted with through the
newly agreed API.

Ledger and Consensus will then be unblocked from beginning to integrate, i.e.
modifying their code to use the new API.

In parallel with this, we will begin implementing a B+-tree. The focus will
be to produce a working solution, not to achieve good performance. We
will verify the correctness of this implementation with io-sim. The intention
will be to use this as a skeleton with with to implement an LSM tree.

Once the B+-tree is feature complete, and Related Components have completed
their integration, we would have the option to tune the performance of the
B+-tree so that it could be deployed into production. Whether we would do this
or not would depend on the urgency of lowering the memory footprint of Cardano
Node.

We will then replace the B+-tree in the previous solution with a bespoke LSM tree.
Our implementation will be written to use batched IO from the outset. We will
verify it's correctness with the test suite developed against the B+tree
implementation. We will iterate on this LSM tree, improving it's performance
as required.

\section{Risks}

Here we call out specific risks we see in the project, and in our preferred option in particular
likely

\subsection{API design}
The design of the API will have far reaching impact on the success of the
project. We need to carefully balance the need to integrate smoothly with Ledger
and Consensus and the requirement that the API can be implemented with Parallel
IO. This points one towards fixing the API early, so that integration in the
related components can start, however this risks increasing the costs, or
rendering impossible,  incorporating learnings from experience implementing the
data structure into the design of the final API.

\subsection{Performance of Chain Sync}
As discussed in \todo{ref section} there is reason to expect that any solution
will have sync times in the order of days per chain-year at even moderate
Transaction rates. This is the critical constraint that guides the majority of
our design. There is some opportunity to mitigate this e.g. \ref{Caching Layer},
but this will remain a looming threat if transaction rates increase
substantially. We recommend that IOHK direct it's researchers to investigate
ways that nodes can sync a chain without having to validate the entire history
of the chain.

\addcontentsline{toc}{section}{References}
\bibliographystyle{plainnat}
\bibliography{utxo-db}

\end{document}
