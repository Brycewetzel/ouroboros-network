\documentclass[11pt,a4paper]{article}

\usepackage[margin=2.5cm]{geometry}
\usepackage[dvipsnames]{xcolor}
\usepackage{todonotes}
\usepackage{microtype}
\usepackage{amssymb,amsmath}
\usepackage{mathpazo}
\usepackage{longtable,booktabs}
\usepackage{dcolumn}
\usepackage{graphicx}
\usepackage{hyperref}
\usepackage[capitalise,noabbrev,nameinlink]{cleveref}
\hypersetup{
  pdftitle={Storing the Cardano ledger state on disk: analysis and design options},
  pdfborder={0 0 0},
  breaklinks=true
}

\begin{document}

\title{Storing the Cardano ledger state on disk: \\
       analysis and design options \\
       {\large \sc An IOHK technical report}
  }
\date{Version 0.9, April 2021}
\author{Douglas Wilson     \\ {\small \texttt{douglas@well-typed.com}} \\
   \and Duncan Coutts      \\ {\small \texttt{duncan@well-typed.com}} \\
                              {\small \texttt{duncan.coutts@iohk.io}}
   }

\maketitle

%\listoftodos

\section{Introduction}
\label{introduction}

The project is intended to solve the following problem: the Cardano node keeps
its ledger state within memory (RAM) and as Cardano scales up this will not be
sustainable because the ledger state will eventually grow too big.

The solution that this project is focused on is to move the bulk of the ledger
state from memory to disk. This will involved developing and then integrating
new infrastructure in the Cardano node (ledger and consensus layers) to allow
large parts of the ledger state to be kept on disk rather than in memory.

This document is intended to capture the business and technical requirements,
to present a small set of candidate solutions, and to evaluate and justify a
preferred solution among those candidates. That solution should be justified in
complexity by the captured requirements.

\tableofcontents

\section{Summary}
\label{summary}

Memory (RAM) is very fast but memory space is relatively small and expensive.
Disks -- even SSDs -- are relatively slow, but disk space is relatively
plentiful and cheap.

Thus the challenge when adapting a design from memory to disk is to maintain
adequate performance. For blockchains generally, and Cardano specifically, this
is not a trivial problem. For example for a long time Ethereum node performance
was bottlenecked on disk I/O performance. We must consider performance in the
design analysis or we will face the same problem.

Best case estimates indicate that the stretch goal of 200 TPS is \emph{very}
hard to achieve if one also wants the time to synchronise the chain to be
reasonable (e.g. the first time Daedalus is used). Even threshold or mid
targets of 20 and 50 TPS will be a challenge, while keeping sync times
reasonable.

Overall this points to the next major scaling bottleneck being synchronisation
performance. Though it is out of scope for this project, it will likely be
worth developing solutions that do not require all nodes (e.g. Daedalus nodes)
to download and validate the entire chain.

The simplified optimistic estimate is as follows:
\begin{itemize}
\item Assume the 200 TPS stretch goal.
\item Assume the typical 2 inputs and 2 outputs per tx.
\item Assume the UTxO is the only part of the ledger state of interest. This is
      a simplifying assumption. In reality there are other parts of the ledger
      state that will make these estimates worse.
\item Assume the UTxO mostly does not fit in memory.
\item Assume the UTxO read access pattern has poor temporal and physical
      locality, and thus each lookup will typically require at least one
      ``random'' disk read.
\item Assume a DB read amplification factor of 1.5. (This is a rosy assumption.)
\item Thus 200 TPS translates to $200 * 2 * 1.5 = 600$ disk I/O operations per
      second (IOPS).
\item Assume a mid-range SSD rated at 10,000 IOPS for random reads at queue
      depth 1, and 100,000 IOPS for random reads at queue depth 32.
\item Assume that we can fully utilise the parallel performance of the SSD,
      i.e. use queue depth 32.
\item The ratio of SSD performance IOPS (100,000) to live system IOPS (600)
      gives the sync speed ratio, i.e. the factor that syncing the chain would
      be compared to real time. This is a factor of $100,000 / 600 = 167$ in
      our example.
\item Thus for a chain growing at 200 TPS for a year, the time to sync that
      chain would be $1/167$ of a year, which is 52.5 hours, more than two days.
\end{itemize}

As we can see in this estimate, the chain sync times are not reasonable.
The example illustrates that the crucial factors are:
\begin{enumerate}
\item The target TPS
\item The read amplification factor
\item The SSD random read performance
\end{enumerate}
Thus if we reduce the target TPS by a factor of 10, we reduce the sync time
correspondingly. With 20 rather than 200 TPS we could expect 5 hours of syncing
to catch up a year of the chain.

A read amplification factor of only 1.5 is itself a challenge, and requires a
good database choice.

Modern SSDs IOPS for random read range from 100,000 to 1,000,000 for the
extreme end of the consumer market. Achieving these levels of performance
requires utilising parallel I/O. If only serial I/O is used, one is limited to
the approximately 10,000 -- 20,000 IOPS. Using parallel I/O requires a more
complex design and developing other additional software infrastructure.

So as we can see, even with 20 TPS, to achieve reasonable sync times will
require a sophisticated choice of database, and the development effort needed
to take advantage of parallel I/O. Or it requires relaxing the assumption that
most of the ledger state does not fit in memory: allowing users with lots of
memory to sync quickly, while other users sync slowly.

The recommended development path is to assume that initially (e.g. 12 months)
the TPS will remain relatively low, and that the size of the ledger state will
remain only somewhat bigger than memory. In this case it may be possible to
develop a solution that does not initially use a very sophisticated database
and does not use parallel I/O, but follows a design that can be extended to do
so.

\section{Requirements and targets}
\label{requirements}


\section{Related components}
\label{components}


\section{Technical constraints}
\label{constraints}


\section{Design options}
\label{options}


\section{Design analysis}
\label{analysis}


\section{Preferred option}
\label{preferred}


%\addcontentsline{toc}{section}{References}
%\bibliographystyle{alpha}
%\bibliography{utxo}

\end{document}
