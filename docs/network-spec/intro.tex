\chapter{Overview}
The Cardano blockchain system is based on the Ouroboros family of protocols
for a proof-of-stake cryptocurrency.
The operation of these protocols on a node, working in collaboration with the consensus
and ledger aspects of Cardano, create the overall distributed system.
It is that distributed system that defines the "single source of truth" that is the distributed ledger.
The Ouroboros papers describe these protocols in a high-level mathematical formalism
that allows for a concise presentation and is appropriate for peer review by cryptography experts.
For separation of concerns, the papers do not describe a concrete implementation of the Ouroboros
protocols.

This document has a broader scope.
It is addressed to system designers and engineers who have an IT background
but are not necessarily crypto experts,
but want to implement Cardano or to understand the reference implementation.
The description of the protocol in this document contains all the information needed to
implement the data diffusion functionality of a compatible Cardano network node.
It covers:
\begin{itemize}
\item How nodes join the network.
\item The general semantics of the messages that nodes exchange.
\item The binary format of the messages.
\item The order in which nodes send and receive the messages of the protocol.
\item The permissible protocol states for each participant in the protocol.
\end{itemize}
This information is typically found in the description of network protocols.

However, the Ouroboros proof-of-stake cryptocurrency has additional requirements,
of a sort that are not typically covered in a
protocol  description itself.
While these underlying requirements are essential for understanding the design of the protocol,
it also makes sense to also discuss these aspects and requirements in this document.
Typical network protocols describe simple information exchanges;
the distributed nature of blockchain computation means
that additional contextual information is available.
Use of this context allows, for example, for validated store and forward
which is an essential feature to contain the effect of potential malicious actions
against the distributed system.

Shelley is a first fully decentralised release of Cardano system, implementing the Ouroboros protocol.

This Chapter contains an overview of the content and scope of this document and aspects that
are being discussed.

\wip{
\subsubsection{Software assurance}
Software assurance
}

\subsubsection{Layered Protocols}
Traditionally network protocols are presented as a stack of layers where
upper layers build on services provided by lower layers and lower layers
are independent of the implementation of the upper layers.
This concept of layers is misleading when discussing Ouroboros.
For example, it is {\em not} the case that the consensus (layer)
is built on top of the network (layer).
It is more appropriate to talk about a network component than a network layer.
The network component provides services to the consensus component and vice versa;
both components rely on each other.
The network component uses the consensus component to validate
information that it distributes to other nodes, which
is essential to guard against certain kinds of DoS attacks.
Existing peer-to-peer systems focus on a slightly different problem domain.
For example, they do not consider the information validation issue
or are concerned with issues such as 'eclipse attacks' that to not
apply to the Ouroboros family of protocols.

%\missingfigure[figwidth=6cm]{Testing a long text string}

\subsubsection{Performance of the Ouroboros Network}
In computer science, Byzantine Fault Tolerance is a property of a distributed algorithm, which states
that it works for the honest participants
under the assumption that a certain proportion of the participants are indeed honest.
A similar, but more informal property applies to performance of the Ouroboros network as well.

The network provides a service to its participants while at the same
time the participants provide a service to the network.
The performance of the Ouroboros network depends (among other things) on the performance of the nodes,
while the performance of a node also depends on the performance of the network.
Not only are there honest and adversarial participant, but there is also a huge variety of
possible network topographies, bandwidths and latencies of network connections and other
factors that determine the performance of the network.

This document discusses the high level functional and performance requirements for Ouroboros and the
assumptions made about the structure of the underlying P2P network.

\subsubsection{Protocol vs Implementation}
Network Protocols are written at different levels of abstraction.
To be useful, a protocol description must be precise enough to be implemented.
A protocol description should also be abstract enough to allow alternative implementations of the protocol
and to facilitate developing and improving it.
Furthermore, it should be possible to both implement an
abstract version of a protocol
and interpret it in a real-world scenario.
For example, it must be possible to implement a protocol
such that the real-world software runs on a machine with a typical size of memory
and a typical speed network connection.

The Shelley network protocol design has been developed in parallel
with a reference implementation in Haskell.
Haskell (more precisely GHC) has built-in support for high level concurrency abstractions
such as light-weight threads and software transactional memory.
While the protocol itself is completely language agnostic, it still makes sense to discuss some
aspects of the Haskell reference implementation in this document.
In particular, the document describes how to achieve good resource bounds for
a protocol implementation.

\wip{
  \subparagraph{Threats}
  Reference the Threats section.
  'eclipse' can be deterred}


\wip{TODO:extended abstract, scope of the document}

\section{High level requirements and User Stories}

These are the high level business requirements for the networking that were
gathered and signed off in late 2017. As such they are expressed in informal
prose, often following a ``user story'' style.
Roughly, there are three different kinds of users:
\begin{itemize}
\item Users who have delegated.
\item Small stakeholders.
\item Large stakeholders.
\end{itemize}

\subsubsection{Network connectivity}\label{network-connectivity}

\paragraph{Participate as a user who has delegated}

As a Daedalus home user with my stake delegated to other users
I would like to join the Cardano network so I can participate in the network.
\begin{itemize}
\item The system must be designed to provide this user segment with the ability
      to catch up and keep up with the blockchain without having
      to do any local network configuration.
\item The system must be designed to provide this user segment with the ability to
      continuously find and maintain a discovery of a sufficient number of
      other network participants that have reasonable connectivity.
\item The system must be designed to provide this user segment with the ability to
      find and maintain a minimum of 3 other network participants to maintain
      connectivity with performance that is sufficient to catch up with the
      blockchain.
\item The system design will take into account that this user will probably be
      behind a firewall.
\item Users in the segment can be defined by having all their stake
      delegated to other network participants.
      As such they will never be selected as a slot leader (i.e required to generate a block).
\end{itemize}


\paragraph{Participate in network as small stakeholder}

As a Daedalus home user operating a node with a small stake,
I would like to join the Cardano network so I can participate in the network as a
node that produces blocks i.e. my stake is not delegated to someone else.

\begin{itemize}
\item The system must be designed to provide this user segment with the ability to
      receive the transactions that will be incorporated into blocks (although
      sizing the operation of the distributed system to ensure that all such
      participants would be able to receive all transactions is not a bounding
      constraint).
\item The system must be designed to provide this user segment with the ability to
      participate in the MPC protocol\footnote{This requirement is now
      redundant because the MPC protocol is specific to Ouroboros Classic.}.
\item The system will be designed to provide this user segment with the ability
      to catch up and keep up with the blockchain without having to do any local
      network configuration (this is a bounding constraint).
\item The user will have sufficient connectivity and performance to receive a
      block within a time slot {\sc and} they have to be able to create and
      broadcast a block within a time slot in which the block is received by
      other participating nodes.
\item The system will be designed to maximise the likelihood that 50\% of home
      users operating a participating node are compliant with the previous requirement
      at any one time.
\item The system will be designed to provide this user segment with the ability
      to continuously find and maintain a discovery of a sufficient number of
      other network participants that have reasonable connectivity.
\item The system will provide a discovery mechanism that will find and maintain
      a minimum of 3 other network participants to maintain connectivity with
      performance that is sufficient to catch up with the blockchain.
\item The system design will take into account that this user may be behind a
      firewall (i.e being behind a firewall should not preclude a user
      participating in this fashion).
\item The Delegation work stream will provide a UI feature for the user to
      choose to control their own stake.
\item Users in this segment will be defined as {\sc not}
      \begin{itemize}
      \item[a)] being in the top 100 users ranked by stake or
      \item[b)] in a ranked set of users who together control 80\% of the stake
      \end{itemize}
\item Users in this segment will not be part of the Core Dif, but still
      subject to the normal incentives related to creating blocks.
\end{itemize}


\paragraph{Participate in network as a large stakeholder}

As a user running a core node on a server and with large stake in the network,
I would like to join the Cardano network so I can participate in the network as
a core server node that produces blocks i.e. have not delegated to someone else.

\begin{itemize}
\item A large stakeholder will be defined as
      \begin{itemize}
      \item[a)] being in the top 100 users ranked by stake; or
      \item[b)] in a ranked set of users who combined control 80\% of the stake
      \end{itemize}
\item Assuming that this user has sufficient connectivity and performance, the
      system should ensure that the collective operation of the distributed
      system will ensure that they have a high probability of receiving a
      block within a time slot such that they have sufficient time to be able
      to create and broadcast a block within a time slot where the block is
      received by other core nodes.
\item It is expected that the previous requirement will be fulfilled to a high
      degree of reliability between nodes in this category -- assuming normal
      network operations

      \begin{tabular}{rl}
      Threshold & $>95\%$ \\
      Target    & $>98\%$ \\
      Stretch   & $>99\%$
      \end{tabular}
\item The system will be designed to provide this user segment with the ability
      to continuously find and maintain a discovery of a sufficient number of
      other network participants that have reasonable connectivity.
\item Discovery will find and maintain a minimum of 10 other network
      participants to maintain connectivity with performance that is sufficient
      to catch up with the blockchain.
\item Ability to receive the transactions that will be incorporated into blocks.
\item Ability to participate in the MPC protocol\footnote{This requirement is
      now redundant because the MPC protocol is specific to Ouroboros Classic.}.
\item The user will catch up and keep up with the blockchain.
\item The server firewall rules will be such that it can communicate with other
      core nodes on the system (and vice versa) -- The system will provide the
      necessary information to update firewall rules if the server is operating
      behind a firewall to ensure the server can communicate with other core
      nodes.
\item The threshold which defines the group of large stakeholders may be
      configurable on the network layer. The configuration may include toggling
      between the rules a) and b) in the previous requirement and the threshold
      numbers within these (this is pending a decision from the Incentives
      work stream.
\item The rules and threshold configuration may need to be a protocol parameter
      that is updated by the update system.
\end{itemize}


\paragraph{Poor network connectivity notification}

As a home user, I want to see a network connection status on Daedalus so that
I know the state of my network connection.
%
\begin{itemize}
\item If the user receives a notification that they are in red or amber mode,
      Daedalus will give the user some helpful information on how to resolve
      common connectivity issues.
\end{itemize}
%
There are three (at least) the following three distinct modes that the network can be operating in:
each one has a red, green, amber status.

%
\begin{center}
\begin{tabular}{ll}
Initial block sync \\
\hline
red   & receiving $<1$ blocks per 10s \\
amber & receiving $<10$ blocks per 10s \\
green & otherwise  \\[1em]

Recovery \\
\hline
red   & receiving $<1$ block per 10s \\
amber & otherwise  \\
green & (not applicable) \\[1em]

Block chain following \\
\hline
red   & it has been more that 200s since a slot indication was received. \\
amber & it has been more than 60s since a slot indication was received. \\
green & otherwise.
\end{tabular}
\end{center}

This assumes that the slot time remains 20 seconds, or at least that the average time
between production of new blocks is 20 seconds.

\paragraph{Transaction Latency}

As a user I want my transaction to be submitted to the blockchain and received
by the target user within the following time period:
%
\begin{center}
\begin{tabular}{lr}
Threshold & 100 seconds \\
Target    & 60 seconds  \\
Stretch   & 30 seconds  \\
\end{tabular}
\end{center}
%
The above time-frames will be achieved for $>95\%$ of all transactions.

\paragraph{Network Bearer Resource Use -- end user control}

As a user operating on the network as a home user not behind a firewall, I
would like a cap on the total amount of network capacity in terms of short-term
bandwidth that other network users can request from my connection so I am
assured my network resource is not eaten up by the data diffusion function.

\begin{itemize}
\item The cap should be based on a fraction of a typical home internet
      connection -- it can be changed by configuration including ``don't act
      as a super node''.
\item The system will allow users syncing with the latest version of the
      blockchain to download blocks from more then one and up to five network
      peers concurrently.
\item A cap on number of incoming subscribers.
\item A cap on number of outbound requests for block syncing from other users.
\item The cap will not be imposed on core nodes running on a server.
\item If these resources are available, a reasonable connection speed should be
      available to users requesting to sync the latest version of the
      blockchain e.g. downloading blocks from 5 peers concurrently to aggregate
      the bandwidth.
\item (nice to have) the actual number and capacity being used is available to
      user.
\end{itemize}

\paragraph{Participant performance measurement}

There may be a requirement for measuring if a large stakeholder is not meeting
their network obligation \cite{DBLP:journals/corr/abs-1807-11218}.

It is accepted that this requirement is a ``nice to have'', and it has not been
established that it is possible, nor has it been incorporated into the
incentives mechanism.

\subsubsection{Distributed System Resilience and Security}

\paragraph{Resilience to abuse}

As a user I should not be able to attack the system using an asymmetric denial
of service attack that will deplete network resources from other users.

\begin{itemize}
\item The system should achieve its connectivity and performance requirements
      even in the presence of a non-trivial proportion of bad actors on the
      network.
\item There is an assumption that there are not a large numbers of bad actors
      in the network.
\item The previous assumption does not follow from the assumptions of Ouroboros
      which states that the users that control 50\% of the stake are
      non-adversarial.
\end{itemize}


\paragraph{DDoS protection}

As a large stakeholder running a core node on a server, I should still be able
to communicate with other user in this segment, even if the system comes under
a DDoS attack.

\begin{itemize}
\item Users in this segment will be able to generate and broadcast blocks to
      each other within the usual timing constraints in this situation.
\end{itemize}

IP addresses will be hidden.

\begin{itemize}
\item Encrypted IP addresses will be published by 10 of the other members of
      the group of large stakeholder core nodes.
\end{itemize}

Assumption

\begin{itemize}
\item Core node operators will not publish their IP addresses publicly.
\item Encrypted IP addresses will be published by the 10 of the other members
      of the group of large stakeholder core nodes.
\item If a node operator's IP address is compromised the operator will respond
      and change the IP address of their node.
\item The system will allow operators to change the address of their core nodes
      and communicate with that new IP address within a reasonable period of
      time.
\end{itemize}


\subsubsection{Network decentralisation}

\paragraph{No hegemony}

As a user I want to be assured that IOHK and its business partners are not in
an especially privileged position in terms of trust, responsibility and
necessity to the network so that network hegemony is avoided.

\begin{itemize}
\item IOHK should be in the same position on the network as any other
      stakeholder with an equivalent amount of stake.
\item There is a more general requirement that no other actor could
      achieve hegemonic control of the operation of the data diffusion layer.
\end{itemize}


\wip{
\section{Context and Introduction}

How this work relates (in general terms) to the rest of the Shelley
development and the Ouroboros papers.

\subsection{Data Diffusion assumptions in Ouroboros}
\begin{quote}
  This is the ``telling them what you are going to tell them'' part -
  outline of the data diffusion and overlay network bringing out the
  key functional and non-functional relationships - aim to serve as a
  general executive overview as well as a framing of PoS issues.
\end{quote}
\begin{itemize}
  \item Assumptions of the mathematical model(s)
  \item Goal of the data diffusion functionality
  \item Strong requirement on collective performance
  \begin{itemize}
    \item Chain growth quality
    \item Adversarial actor assumption
  \end{itemize}
\end{itemize}
\subsection{Functional Layering}
\hide[inline]{pictures as to how the various functional layers relate; How the Data
diffusion layer relates to Ledger etc; How data diffusion relates to
point-to-point overlay network.}

\subsubsection{Non-function aspects}
Performance; trustworthiness; Forwarding as an expression of confirmed
``trust'' (and corollary - forwarding of \emph{clearly} incorrect information
seen as prima-facie evidence of adversarial action.
\subsection{Protocol Roles}
Peer relationship between various nodes; Limited trust and
verification; (Brief) description of the expectations and assumptions across
the functionality boundaries;
}

\wip{
\subsubsection{Mini Protocols}
\begin{description}
\item[Chain Sync]
\item[Block Fetch]
\item[Transaction Submission]
\item[$\Delta Q$ Measurement] (not really a mini protocol, in that it
  is point-to-point and not part of the diffusion process itself,
  placed here because it ``sits'' on top of the overlay network. Role
  to generate active endpoint performance data to help optimise
  time-to-diffuse critical information exchanges (e.g. newly minted
  block diffusion)
\end{description}
}
\wip{
\subsubsection{Point-to-point Overlay Network}
\hide[inline]{need a picture} Note that long term eclipse attacks are not an issue
here (cover why a bit later); tie in chain growth requirements with
need for ``better'' communications performance between (major)
stake pools (notion of Core DIF); association explain need for
\begin{itemize}
\item fixed configuration (with/without others from this list)
\item Core DIF
\item Distributed endpoint discovery (subset of notion of peer in
  other approaches)
\end{itemize}
}

\wip{
\section{Layout of the Document}

\begin{itemize}
  \item What goes in which section ?
  \item In which order to read ?
  \item Which sections can be skipped ?
\end{itemize}

\section{Notation}
}

\wip{
\chapter{Requirements}

\section{Performance Requirements and User Stories}
\subsection{Classes of Participants}
\wip{todo: make a nice table}
\hide{Some information about participants is in the previous
section~\ref{network-connectivity}}

\subsubsection{Stake pool} % lookup what this is called in the protocols.tex
\subsubsection{Small stakeholder}
\subsubsection{User who has delegated}
\subsubsection{Requirements for Participants}
\subsubsection{Requirements for Stake Pools}
\subsubsection{Services that the System should provide}

There are two kinds of Requirements:

\begin{enumerate}
\item System capabilities for a node to take a blockchain slot creation role in the protocol.
\item What services that the system provides to the user.
\end{enumerate}

}

\hide{
\section{Protocol Updates on the Blockchain}
\begin{itemize}
\item Hybrid phase of federation and decentralisation
\item Gradually transition between protocol variants on a live blockchain.
\item Several protocol variants active in parallel, version negotiation.
\item Communication between Shelley Nodes and existing core nodes.
\item Byron proxy
\end{itemize}

\section{Node to Node and Node to Consumer IPC}
There are two basic variants of inter-process-communication in the network:
\begin{itemize}
\item IPC between Cardano nodes that are engaged in the high level Ouroboros
      blockchain consensus protocol.
\item IPC between a Cardano node and a `chain consumer' component such as a
      wallet, explorer or other custom application.
\end{itemize}
Both variants of IPC in the network follow distinct requirements and constraints, and
,while the first version of Cardano used a single protocol, the new version will
use different sets of protocols for both uses cases.
(See Section~\ref{why_distinguish_protocols} for the motivation for this design decision.)
Throughout the document it will be clear which variant of we are referring to.

\section{Threat Model}

\wip{
  WIP discussion of the threat model
\subsection{Resource Consumption Attacks}
}
}
\hide{
\section{Ouroboros}
\wip{
How PoS is different from PoW in its network requirements:
}

\begin{itemize}
  \item No capability to sustain an undetected Eclipse attack
  \item Sustained liveness requirement
\end{itemize}

\section{Delegation}
}

\chapter{System Architecture}
\hide{
\section{Overview}
This section describes the key underlying aspects and considerations of the design
of the network protocol and tries to give a big picture the overall network protocol.

\section{Key Functionality}

\wip{
  Here only an outline. Full discussion in the Section~\ref{ouroboros}.
\begin{description}
\item[Block diffusion]
\item[Chain following]
\item[Forwarding transactions]
\end{description}
}

\section{High Assurance Software}
\wip{How does the software/protocol design reflect the high assurance requirements ?}

\section{Modular Design and Mini Protocols}
\wip{Why are mini protocols a good idea ?}
\wip{How does polymorphism help modularity ?}
}

\section{Congestion Control}
A central design goal of the system is robust operation at high workloads.
For example, it is a normal working condition of the networking design
that transactions arrive at a higher rate than the number
that can be included in blockchain.
An increase of the rate at which transactions are submitted must not cause a decrease
of the block chain quality.

Point-to-point TCP bearers do not deal well with overloading.
A TCP connection has a certain maximal bandwidth,
i.e. a certain maximum load that it can handle relatively reliably under normal conditions.
If the connection is ever overloaded,
the performance characteristics will degrade rapidly unless
the load presented to the TCP connection is appropriately managed.


At the same time, the node itself has a limit on the rate at which it can process data.
In particular, a node may have to share its processing power with other processes that run on the
same machine/operation system instance, which means that a node may get slowed down for some reason,
and the system may get in a situation
where there is more data available from the network than the node can process.
The design must operate appropriately in this situation and recover form transient conditions.
In any condition, a node must not exceed its memory limits,
that is there must be defined limits, breaches of which being treated like protocol violations.

Of cause it makes no sense if the system design is robust,
but so defensive that it fails to meet performance goals.
An example would be a protocol that never transmits a message unless it has received an
explicit ACK for the previous message. This approach might avoid overloading the network,
but would waste most of the potential bandwidth.

\hide{
\subsection{Back Pressure}
Back Pressure is a strategy for congestion control for push based buffered channels like
,for example, TCP connections.



\wip{TODO: explain the concept of Back pressure and how it is used in the design}
\subsection{Push vs Pull}
\wip{TODO: data flow inside the node is pull based}
\subsection{Delta Q}
\wip{TODO: explain what delta Q means and what it has to do with congestion control.}
}

\begin{figure}[ht]
  \pgfdeclareimage[height=15cm]{node-diagram-chains-state}{figure/node-diagram-chains-state.pdf}
  \begin{center}
    \pgfuseimage{node-diagram-chains-state}
  \end{center}
  \caption{Data flow inside a Node}
  \label{node-diagram-concurrency}
\end{figure}

\section{Data Flow in a Node}
Nodes maintain connections with the peers that have been chosen
with help of the peer selection process.
Suppose node $A$ is connected to node $B$.
The Ouroboros protocol schedules a node $N$ to generate a new block in a given time slot.
Depending on the location of nodes $A$, $B$ and $N$ in the network topology and whether the new
block arrives first at $A$ or $B$, $A$ can be either up-stream or down-stream of $B$.
Therefore, node $A$ runs an instance of the client side of the chain-sync mini protocol
that talks with a server instance of chain-sync at node $B$ and also a server instance of chain sync
that talks with a client instance at $B$.
The situation is similar for the other mini protocols (block fetch, transaction submission, etc).
The set of mini protocols that runs over a connection is determined by the version of the network
protocol, i.e.  Node-to-Node, Node-to-Wallet and Node-to-Chain-Consumer
connections use different sets of mini protocols (e.g. different protocol
versions).  The version is negotiated when a new connection is established
using protocol which is described in Chapter~\ref{connection-management}.
\hide{Add description of this protocol in Chapter~\ref{connection-management}
and link it.}

Figure~\ref{node-diagram-concurrency} illustrates parts of the data flow in a node.
Circles represents a thread that runs one of the mini protocols (the mini protocols are explained in
Chapter~\ref{chapter:mini-protocols}).
There are two kinds of data flows:
mini protocols communicate with mini protocols of other nodes by sending and receiving messages;
and, within a node, they communicate by reading from- and writing to- a shared
mutable state (represented by boxes in Figure~\ref{node-diagram-concurrency}).
\href{https://en.wikipedia.org/wiki/Software_transactional_memory}{Software transactional memory}
(STM) is a mechanism for safe and lock-free concurrent
access to mutable state and the reference implementation makes intensive use of this abstraction.

\section{Real-time Constraints and Coordinated Universal Time}
Ouroboros models the passage of physical time as an infinite sequence of time slots,
i.e. contiguous, equal-length intervals of time,
and assigns slot leaders (nodes that are eligible to create a new block) to those time slots.
At the beginning of a time slot, the slot leader selects the block chain and transactions that are the basis
for the new block, then it creates the new block and sends the new block to its peers.
When the new block reaches the next block leader before the beginning of next time slot,
the next block leader can extend the block chain upon this block (if the block
did not arrive on time the next leader will create a new block anyway).

There are some trade-offs when choosing the slot time that is used for the protocol but
basically the slot length should be long enough such that a new block has a good chance to reach the
next slot leader in time.
A chosen value for the slot length is 20 seconds.
It is assumed that the clock skews between the local clocks of the nodes is small with respect to the
slot length.

However, no matter how accurate the local clocks of the nodes are with respect to the time slots
the effects of a possible clock skew must still be carefully considered.
For example, when a node time-stamps incoming blocks with its local clock time, it may encounter
blocks that are created in the future
with respect to the local clock of the node.
The node must then decide whether this is because of a clock skew or whether the node considers this
as adversarial behaviour of an other node.

\wip{TODO :: get feedback from the researchers on this. Tentative policy: allow 200ms to 1s
explain the problem in detail.
A node cannot forward a block from the future.
}
