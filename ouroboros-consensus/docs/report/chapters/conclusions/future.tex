\chapter{Future work}

\section{On abstraction}

ledger integration: as things were changing a lot, it made sense for consensus to define the ledger API internally and have the integration be done consensus side. but as things are stabilising, it might make more sense for that abstraction to live externally, so that you can literally plug in Shelley into consensus and we don't have to do anything

\section{Genesis}
\label{future:genesis}

The genesis rule is an improved chain selection rule, designed to help new nodes
safely join the network even in the presence of potential long range attacks
\cite{cryptoeprint:2018:378}. From the point of view of the consensus layer,
however, the \emph{why} is less relevant than the \emph{what} and the
\emph{how}. Let's start with the \emph{what}. The genesis chain selection
rule defined as follows.

\begin{definition}[Genesis chain selection rule]
A candidate chain is preferred over our current chain if

\begin{itemize}
\item The intersection between the candidate chain and our chain is no more
than $k$ blocks back, and the candidate chain is strictly longer than our
chain.

\item If the intersection \emph{is} more than $k$ blocks back, and the
candidate chain is \emph{denser} (contains more blocks) than our chain in
a region of $s$ slots starting at the intersection.
\end{itemize}
\end{definition}

Part (2) if this definition of course is what sets the genesis rule apart
from the Praos chain selection rule.\footnote{The \lstinline!TPraos! chain
selection rule of the Shelley ledger may additionally prefer candidate chains
of equal length under certain circumstances, see \cref{todo}.} It is problematic
for the design of the consensus layer, for at least two reasons:

\begin{itemize}
\item It does not fit the look-at-tip-only approach
(\cref{consensus:overview:chainsel})
\item It violates the ``we never roll back to a shorter chain'' assumption
(\cref{never-shrink})
\end{itemize}






\section{On-disk ledger state}

\duncan

Sketch out what we think it could look like
Consequences for the design

\section{Transaction TTL}
\label{future:ttl}

Describe that the mempool could have explicit support for TTL, but that right now we don't (and why this is OK: the ledger anyway checks tx TTL). We should discuss why this is not an attack vector (transactions will either be included in the blockchain or else will be chucked out because some of their inputs will have been used).

\section{Block based versus slot based}
\label{future:block-vs-slot}

\section{Eliminating safe zones}
\label{future:eliminating-safezones}

Are they really needed? Consensus doesn't really look ahead anymore?
(Headers are not checked for time; leadership is ticking, not forecasting).
Does the wallet really need it? What about the ledger?

Other thought: what if we split slots into "microslots", 20 microslots to a
slot. Now the slot/time mapping is \emph{always} known, and for Shelley etc
we don't actually need to know the global microslot, all we care about is
the microslot within a slot (and hence is independent of when Shelley starts).
This would make time conversion no longer state dependent.

\section{Eliminating forecasting}
\label{future:eliminating-forecasting}

This is a stronger version of \cref{future:eliminating-safezones}, where
we eliminate \emph{all} forecasting. Specifically, this means that we don't
do header validation anymore, relying on the chain DB to do block validation.
This would be an important simplification of the consensus layer, but we'd
need to analyse what the ``benefit'' of this simplification is for an
attacker. Personally, I think it'll be okay.

The most important analysis we need to do here is how this affects the memory
usage of the chain sync client. Note that we already skip the ahead-of-time
check, which we don't do until we have the full block and validate it. We
should discuss that somewhere as well.

\section{Open kinds}
\label{future:openkinds}

Avoid type errors such as trying to apply a ledger to a block instead of an era
(or an era instead of crypto, or..).

\section{Relax requirements on time conversions}
\label{future:relax-time-requirements}

Perhaps it would be okay if time conversions we strictly relative to a ledger
state, rather than ``absolute'' (\cref{time:ledgerrestrictions}).

\section{Configuration}

What a mess.

\section{Specialised chain selection data structure}

In \cref{chainsel:spec} we describe how chain selection is implemented. However,
in an ideal world this would mean we have some kind of specialised data
structure supporting

\begin{itemize}
\item Efficient insertion of new blocks
\item Efficient computation of the best chain
\end{itemize}

It's however not at all clear what such a data structure would look like if we
don't want to hard-code the specific chain selection rule.

\section{Dealing with clock changes}
\label{future:clockchanges}

When the user changes their system clock, blocks that we previously adopted
into our current chain might now be ahead of the system clock (\cref{chainsel:infuture}) and should not
be part of the chain anymore, and vice versa.

When the system clock of a node is moved \emph{forward}, we should run chain
selection again because some blocks that we stored because they were in the
future may now become valid. Since this could be any number of blocks, on any
fork, probably easiest to just do a full chain selection cycle (starting from
the tip of the immutable database).

When the clock is moved \emph{backwards}, we may have accepted blocks that we
should not have. Put another way, an attacker might have taken advantage of the
fact that the clock was wrong to get the node to accept blocks in the future. In
this case we therefore really should rollback--- but this is a weird kind of
rollback, one that might result in a strictly smaller current chain. We can only
do this by re-initialising the chain DB from scratch (the ledger DB does not
support such rollback directly). Worse still, we have have decided that some
blocks were immutable which really weren't.

Unlike the data corruption case, here we should really endeavour to get to a
state in which it was as if the clock was never ``wrong'' in the first place;
this may mean we might have to move some blocks back from the immutable DB to
the volatile DB, depending on exactly how far the clock was moved back and how
big the overlap between the immutable DB and volatile DB is.

It is therefore good to keep in mind that the overlap between the immutable DB
and volatile DB does make it a bit easier to deal with relatively small clock
changes; it may be worth ensuring that, say, the overlap is at least a few days
so that we can deal with people turning back their clock a day or two without
having to truncate the immutable database. Indeed, in a first implementation,
this may be the \emph{only} thing we support, though we will eventually have to
lift that restriction.

Right now, we do nothing special when the clock moves forward (we will discover
discover the now valid blocks on the next call to \lstinline!addBlock!
(\cref{chainsel:addblock}). When the clock is reset \emph{backwards}, the node
will currently (intentionally) crash, we make no attempt to try and reset
the state (the current slot number moving backwards might cause difficulties
in many places). Unfortunately, if the clock is moved so far back that blocks
in the \emph{immutable database} are now considered to be ahead of the wall
clock, we will not currently detect this (\cref{time:imm-tip-in-future}).
